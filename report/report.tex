\documentclass{article}

\usepackage[italian]{babel}
\usepackage[hidelinks]{hyperref}
\usepackage{graphicx}
\usepackage{listings}
\usepackage{xcolor}
\usepackage{mfirstuc}

\definecolor{codegreen}{rgb}{0,0.6,0}
\definecolor{codegray}{rgb}{0.5,0.5,0.5}
\definecolor{codepurple}{rgb}{0.58,0,0.82}
\definecolor{backcolour}{rgb}{0.92,0.92,0.92}

\lstdefinestyle{mystyle}{
	backgroundcolor=\color{backcolour},
	commentstyle=\color{codegreen},
	keywordstyle=\color{blue},
	numberstyle=\tiny\color{codegray},
	stringstyle=\color{codepurple},
	basicstyle=\ttfamily\footnotesize,
	breakatwhitespace=false,
	breaklines=true,
	captionpos=b,
	keepspaces=true,
	numbers=left,
	numbersep=5pt,
	showspaces=false,
	showstringspaces=false,
	showtabs=false,
	tabsize=2
}

\lstset{style=mystyle}

\graphicspath{ {./images/} }

\newcommand{\sqlinputlisting}[2]{
	\lstinputlisting[language=SQL, captionpos=t, caption={#2}]{#1}
}

% File name without extension
\newcommand{\sql}[1]{
	\sqlinputlisting{../src/#1.sql}
	{\texttt{#1.sql}}
}

\title{
	faDBula \\
	\textbf{\large
		Relazione del progetto per l'insegnamento di \break
		Basi di dati
	}
}

\author{
	Ilaria Volpe (\#766012,
	\href{mailto:stefano.volpe2@studio.unibo.it}{stefano.volpe2@studio.unibo.it}),
	\\
	Stefano Volpe (\#969766,
	\href{mailto:ilaria.volpe2@studio.unibo.it}{ilaria.volpe2@studio.unibo.it})
}

\date{
	Alma Mater Studiorum - Universit\`a di Bologna \\
	\today
}

\begin{document}

\maketitle
\thispagestyle{empty}
\tableofcontents

\section{Analisi dei requisiti}

\subsection{Requisiti espressi in linguaggio naturale}

Si desidera progettare una base di dati per gli appassionati di una specifica
produzione narrativa letteraria, cinematografica e/o televisiva. Lo scopo è
quello di permettere una facile analisi di trame anche complesse. Si intende
rappresentarne con codice univoco agenti, tempi, luoghi, eventi della fabula,
nonché unità narrative (anacronie e non) dell'intreccio. Sono memorizzati nome,
immagine descrittiva, sesso, momenti di nascita e di morte degli agenti. I
tempi sono rappresentati da intervalli (lineari) o da fasi (di un ciclo): dei
primi sono memorizzati l'istante di inizio e l'istante di fine mentre delle
seconde, il cui codice è progressivo, il nome. I luoghi sono caratterizzati da
nome e coordinate su una data mappa. Le mappe sono descritte da nome, immagine
ed estensione. Gli eventi sono caratterizzati da veridicità, agenti coinvolti,
luogo e intervallo. Uno o più agenti, per un dato intervallo di tempo, possono
pensare che un evento sia avvenuto. Le unità narrative sono descritte da indice,
titolo e intervallo narrato.

\subsection{G}

\subsection{St}

\subsection{Sp}

\section{Codifica SQL}

\subsection{Definizione dello schema}

\sql{schema}

\subsection{Codifica delle operazioni}

% TODO: \sql{operazione}, dove per ogni operazione.sql (non schema.sql) in src/

\end{document}
