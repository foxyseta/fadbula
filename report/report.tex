\documentclass{article}
\usepackage[italian]{babel}
\usepackage[hidelinks]{hyperref}
\usepackage{graphicx}
\usepackage{calc}
\graphicspath{ {./images/} }

\title{
	faDBula \\
	\textbf{\large
		Relazione del progetto per l'insegnamento di \break
		Basi di dati
	}
}

\author{
	Ilaria Volpe (\#766012,
	\href{mailto:ilaria.volpe2@studio.unibo.it}{ilaria.volpe2@studio.unibo.it}),
	\\
	Stefano Volpe (\#969766,
	\href{mailto:stefano.volpe2@studio.unibo.it}{stefano.volpe2@studio.unibo.it})
}

\date{
	Alma Mater Studiorum - Universit\`a di Bologna \\
	\today
}

\begin{document}

\maketitle
\thispagestyle{empty}
\pagebreak
\tableofcontents
\pagebreak
\section{Analisi dei requisiti}

\subsection{Requisiti espressi in linguaggio naturale}

Si desidera progettare una base di dati per gli appassionati di una specifica
produzione narrativa letteraria, cinematografica e/o televisiva. Lo scopo è
quello di permettere una facile analisi di trame anche complesse. Si intende
rappresentarne con codice univoco agenti, tempi, luoghi, eventi della fabula,
nonché unità narrative (anacronie e non) dell'intreccio. Sono memorizzati nome,
immagine descrittiva, sesso, momenti di nascita e di morte degli agenti. I
tempi sono rappresentati da intervalli (lineari) o da fasi (di un ciclo): dei
primi sono memorizzati l'istante di inizio e l'istante di fine mentre delle
seconde, il cui codice è progressivo, il nome. I luoghi sono caratterizzati da
nome e coordinate su una data mappa. Le mappe sono descritte da nome, immagine
ed estensione. Gli eventi sono caratterizzati da veridicità, agenti coinvolti,
luogo e intervallo. Uno o più agenti, per un dato intervallo di tempo, possono
pensare che un evento sia avvenuto. Le unità narrative sono descritte da indice,
titolo e intervallo narrato.

\subsection{Glossario dei termini}

\begin{center}\begin{tabular}{|p{0.2\textwidth-2\tabcolsep-1.2\arrayrulewidth}|p{0.55\textwidth-2\tabcolsep-1.2\arrayrulewidth}|p{0.22\textwidth-2\tabcolsep-1.2\arrayrulewidth}|}
		\hline
		\textbf{Termine} & \textbf{Descrizione}                                                                                & \textbf{Collegamenti} \\
		\hline
		Agente           & Personaggio o alias di un personaggio presente nella produzione narrativa                           & Evento                \\
		\hline
		Tempo            & Periodo temporale che occorre una o più volte nella produzione narrativa                            & Evento                \\
		\hline
		Intervallo       & Tempo di tipo lineare                                                                               & Evento                \\
		\hline
		Fase             & Tempo che ricorre una o più volte, ciclicamente                                                     & Evento                \\
		\hline
		Luogo            & Luogo presente nella produzione narrativa in cui possono avvenire eventi                            & Evento, Mappa         \\
		\hline
		Mappa            & Rappresentazione grafica in cui collocare uno o più luoghi                                          & Luogo                 \\
		\hline
		Evento           & Ciò che avviene o è pensato essere avvenuto nella produzione narrativa                              & Agente, Luogo, Tempo  \\
		\hline
		Credenza         & L'atto, da parte di uno o più agenti, di pensare che un evento sia avvenuto                         & Agente, Evento        \\
		\hline
		Unità narrativa  & Unità che compongono la produzione narrativa e includono parte, uno o più tempi                     & Tempo                 \\
		\hline
		Fabula           & Elenco degli eventi ordinati cronologicamente rispetto al momento in cui avvengono nella narrazione & Tempo, Intreccio      \\
		\hline
		Intreccio        & Elenco degli eventi ordinati cronologicamente rispetto al momento in cui sono narrati               & Tempo, Fabula         \\
		\hline
	\end{tabular}\end{center}

\subsection{Strutturazione dei requisiti}

\begin{itemize}
	\item \textbf{Frasi di carattere generale}
	      Si desidera progettare una base di dati per gli appassionati di una
	      specifica produzione narrativa letteraria, cinematografica e/o
	      televisiva. Lo scopo è quello di permettere una facile analisi di trame
	      anche complesse. Si intende rappresentarne con codice univoco agenti,
	      tempi, luoghi, eventi della fabula, nonché unità narrative (anacronie e
	      non) dell'intreccio.
	\item \textbf{Frasi relative agli agenti:}
	      Sono memorizzati nome, immagine descrittiva, sesso, momenti di nascita e
	      di morte degli agenti.
	\item \textbf{Frasi relative ai tempi:}
	      I tempi sono rappresentati da intervalli (lineari) o da fasi (di un
	      ciclo): dei primi sono memorizzati l'istante di inizio e l'istante di
	      fine mentre delle seconde, il cui codice è progressivo, il nome.
	\item \textbf{Frasi relative ai luoghi:}
	      I luoghi sono caratterizzati da nome e coordinate su una data mappa.
	\item \textbf{Frasi relative alle mappe:}
	      Le mappe sono descritte da nome, immagine ed estensione.
	\item \textbf{Frasi relative agli eventi:}
	      Gli eventi sono caratterizzati da veridicità, agenti coinvolti, luogo e
	      intervallo. Uno o più agenti, per un dato intervallo di tempo, possono
	      pensare che un evento sia avvenuto.
	\item \textbf{Frasi relative alle unità narrative:}
	      Le unità narrative sono descritte da indice, titolo e intervallo
	      narrato.
\end{itemize}

\subsection{Specifica operazioni}
Si scelgono le frequenze assumendo di tracciare una produzione narrativa
televisiva.
\begin{enumerate}
	\item Inserire un nuovo agente (in media 8 volte al mese)
	\item Inserire un nuovo intervallo (in media 80 volte al mese)
	\item Inserire una nuova fase (in media 2 volte al mese)
	\item Inserire un nuovo luogo (in media 7 volte al mese)
	\item Inserire una nuova mappa (in media 1 volta al mese)
	\item Inserire un nuovo evento(in media 160 volte al mese)
	\item Inserire una nuova unità narrativa (in media 4 volte al mese)
	\item Dato un personaggio, visualizzarne tutti gli alias (in media 8 volte al
	      mese)
	\item Dato un istante, visualizzare tutti gli agenti indicando se in vita o
	      meno(in media 4 volte al mese)
	\item Dato un luogo, visualizzare tutti quelli presenti nella stessa mappa (in
	      media 7 volte al mese)
	\item Dato un personaggio, visualizzare tutte le sue false credenze ordinate
	      cronologicamente (in media 1 volta al mese)
	\item Dato un personaggio e un evento, visualizzare eventuali intervalli di
	      tempo in cui il primo crede che sia avvenuto il secondo (in media 4
	      volte al mese)
	\item Visualizzare la fabula (in media 1 volta al mese)
	\item Visualizzare l'intreccio (in media 1 volta al mese)
\end{enumerate}


\section{Progettazione concettuale}

\subsection{Identificazione delle entità e relazioni (bottom-up)}
Si identificano le seguenti entità: personaggio, alias, intervallo,
fase, luogo, mappa, evento e unità narrativa.
Le prime due entità sono generalizzabili nell'entità agente.
\subsection{Un primo scheletro dello schema (top-down)}
% \begin{figure}
%   \includegraphics[width=0.5\linewidth]{file.eps} % assumendo che file.eps sia dentro images/
% \end{figure}
Un primo scheletro di schema concettuale è composto da quattro entità (agente,
intervallo, luogo, evento) e quattro relazioni. La relazione \emph{chi}
specifica quale agente partecipi all'evento, la relazione \emph{quando}
specifica a quale intervallo o fase appartenga l'evento, la relazione
\emph{dove} specifica in quale luogo avvenga l'evento e la relazione
\emph{credenza} descrive l'agente che crede all'evento per l'intervallo.
\subsection{Sviluppo delle componenti dello scheletro (inside-out)}

\subsection{Unione delle componenti nello schema finale ridotto}

\subsection{Dizionario dei dati}

\begin{center}\begin{tabular}{ |c|c|c|c| }
		\hline
		\textbf{Nome entità} & \textbf{Descrizione} & \textbf{Attributi} & \textbf{Identificatore} \\
		\hline
		cell5                & cell6                & cell7              & cell8                   \\
		\hline
		cell9                & cell10               & cell11             & cell12                  \\
		\hline
	\end{tabular}\end{center}

\begin{center}\begin{tabular}{ |c|c|c|c| }
		\hline
		\textbf{Nome relazione} & \textbf{Descrizione} & \textbf{Entità coinvolte} & \textbf{Attributi} \\
		\hline
		cell5                   & cell6                & cell7                     & cell8              \\
		\hline
		cell9                   & cell10               & cell11                    & cell12             \\
		\hline
	\end{tabular}\end{center}


\subsection{Regole}
\begin{center}\begin{tabular}{ |c| }
		\hline
		\textbf{Regole di vincolo}     \\
		\hline
		cell5                          \\
		\hline
		cell9                          \\
		\hline
		\textbf{Regole di derivazione} \\
		\hline
		cell5                          \\
		\hline
		cell9                          \\
		\hline
	\end{tabular}\end{center}

\section{Progettazione logica}

\subsection{Tavole dei volumi e delle operazioni}
\subsubsection{Tavola dei volumi}

\begin{center}\begin{tabular}{ |c|c|c| }
		\hline
		\textbf{Concetto} & \textbf{Tipo} & \textbf{Volume} \\
		\hline
		cell5             & cell6         & cell7           \\
		\hline
		cell9             & cell10        & cell11          \\
		\hline
	\end{tabular}\end{center}

\subsubsection{Tavola delle operazioni}

\begin{center}\begin{tabular}{ |c|c| }
		\hline
		\textbf{Operazione} & \textbf{Frequenza} \\
		\hline
		cell5               & cell6              \\
		\hline
		cell9               & cell10             \\
		\hline
	\end{tabular}\end{center}

\subsection{Ristrutturazione dello schema concettuale}

\subsection{Normalizzazione}
Analizzando lo schema concettuale si nota che tutte le associazioni presenti
sono in forma normale di Boyce e Codd perché ???

\subsubsection{Entità}

\begin{center}\begin{tabular}{ |c|c| }
		\hline
		\textbf{Nome entità} & \textbf{Commento} \\
		\hline
		cell5                & cell6             \\
		\hline
		cell9                & cell10            \\
		\hline
	\end{tabular}\end{center}

\subsection{Traduzione verso il modello relazionale}

\begin{center}\begin{tabular}{ |c|c| }
		\hline
		\textbf{Entità-Relazione} & \textbf{Traduzione} \\
		\hline
		cell5                     & cell6               \\
		\hline
		cell9                     & cell10              \\
		\hline
	\end{tabular}\end{center}

\begin{center}\begin{tabular}{ |c|c| }
		\hline
		\textbf{Traduzione} & \textbf{Vincoli di riferimento} \\
		\hline
		cell5               & cell6                           \\
		\hline
		cell9               & cell10                          \\
		\hline
	\end{tabular}\end{center}


\section{Codifica SQL}

\subsection{Definizione dello schema}

\subsection{Codifica delle operazioni}

\section{Testing}

\end{document}
