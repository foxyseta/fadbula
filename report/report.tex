\documentclass{article}
\usepackage[italian]{babel}

\title{
  faDBula \\
  \textbf{\large
      Relazione del progetto per l'insegnamento di \break
      Basi di dati
  }
}

\author{
  Ilaria Volpe (\#766012),
  Stefano Volpe (\#969766)
}

\date{
	Alma Mater Studiorum - Universit\`a di Bologna \\
  \today
}

\begin{document}

\maketitle

\section{Analisi dei requisiti}

\subsection{Requisiti espressi in linguaggio naturale}

Si desidera progettare una base di dati per gli appassionati di una specifica
produzione narrativa letteraria, cinematografica e/o televisiva. Lo scopo è
quello di permettere una facile analisi di trame anche complesse. Si intende
rappresentarne con codice univoco agenti, tempi, luoghi, eventi della fabula,
nonché unità narrative (anacronie e non) dell'intreccio. Sono memorizzati nome,
immagine descrittiva, sesso, momenti di nascita e di morte degli agenti. I
tempi sono rappresentati da intervalli (lineari) o da fasi (di un ciclo): dei
primi sono memorizzati l'istante di inizio e l'istante di fine mentre delle
seconde, il cui codice è progressivo, il nome. I luoghi sono caratterizzati da
nome e coordinate su una data mappa. Le mappe sono descritte da nome, immagine
ed estensione. Gli eventi sono caratterizzati da veridicità, agenti coinvolti,
luogo e intervallo. Uno o più agenti, per un dato intervallo di tempo, possono
pensare che un evento sia avvenuto. Le unità narrative sono descritte da indice,
titolo e intervallo narrato.
\end{document}