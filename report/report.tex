\documentclass{article}
\usepackage[italian]{babel}
\usepackage[hidelinks]{hyperref}
\usepackage{graphicx}

\graphicspath{ {./images/} }

\title{
	faDBula \\
	\textbf{\large
		Relazione del progetto per l'insegnamento di \break
		Basi di dati
	}
}

\author{
	Ilaria Volpe (\#766012,
	\href{mailto:stefano.volpe2@studio.unibo.it}{stefano.volpe2@studio.unibo.it}),
	\\
	Stefano Volpe (\#969766,
	\href{mailto:ilaria.volpe2@studio.unibo.it}{ilaria.volpe2@studio.unibo.it})
}

\date{
	Alma Mater Studiorum - Universit\`a di Bologna \\
	\today
}

\begin{document}

\maketitle
\thispagestyle{empty}
\tableofcontents

\section{Analisi dei requisiti}

\subsection{Requisiti espressi in linguaggio naturale}

Si desidera progettare una base di dati per gli appassionati di una specifica
produzione narrativa letteraria, cinematografica e/o televisiva. Lo scopo è
quello di permettere una facile analisi di trame anche complesse. Si intende
rappresentarne con codice univoco agenti, tempi, luoghi, eventi della fabula,
nonché unità narrative (anacronie e non) dell'intreccio. Sono memorizzati nome,
immagine descrittiva, sesso, momenti di nascita e di morte degli agenti. I
tempi sono rappresentati da intervalli (lineari) o da fasi (di un ciclo): dei
primi sono memorizzati l'istante di inizio e l'istante di fine mentre delle
seconde, il cui codice è progressivo, il nome. I luoghi sono caratterizzati da
nome e coordinate su una data mappa. Le mappe sono descritte da nome, immagine
ed estensione. Gli eventi sono caratterizzati da veridicità, agenti coinvolti,
luogo e intervallo. Uno o più agenti, per un dato intervallo di tempo, possono
pensare che un evento sia avvenuto. Le unità narrative sono descritte da indice,
titolo e intervallo narrato.

\subsection{Glossario dei termini}

\begin{tabular}{ |c|c|c|c| }
	\hline
	\textbf{Termine} & \textbf{Descrizione}                                                            & \textbf{Sinonimi} & \textbf{Collegamenti} \\
	\hline
	Agente           & Personaggio o alias di un personaggio presente nella produzione narrativa       & -                 & Evento                \\
	\hline
	Tempo            & Periodo temporale in cui avvengono uno o più eventi                             & -                 & Evento                \\
	\hline
	Intervallo       & Tempo di tipo lineare                                                           & -                 & Evento                \\
	\hline
	Fase             & Tempo di tipo ciclico                                                           & -                 & Evento                \\
	\hline
	Luogo            & Luogo presente nella produzione narrativa in cui possono avvenire eventi        & -                 & Evento, Mappa         \\
	\hline
	Mappa            & Strumento per localizzare un luogo                                              & -                 & Luogo                 \\
	\hline
	Evento           & Ciò che avviene o è pensato essere avvenuto nella produzione narrativa          & -                 & Agente, Luogo, Tempo  \\
	\hline
	Unità narrativa  & Unità che compongono la produzione narrativa e includono parte, uno o più tempi & -                 & Tempo                 \\
	\hline
\end{tabular}

\subsection{Strutturazione dei requisiti}

\begin{itemize}
	\item \textbf{Frasi di carattere generale:}
	      Si desidera progettare una base di dati per gli appassionati di una
	      specifica produzione narrativa letteraria, cinematografica e/o
	      televisiva. Lo scopo è quello di permettere una facile analisi di trame
	      anche complesse. Si intende rappresentarne con codice univoco agenti,
	      tempi, luoghi, eventi della fabula, nonché unità narrative (anacronie e
	      non) dell'intreccio.
	\item \textbf{Frasi relative agli agenti:}
	      Sono memorizzati nome, immagine descrittiva, sesso, momenti di nascita e
	      di morte degli agenti.
	\item \textbf{Frasi relative ai tempi:}
	      I tempi sono rappresentati da intervalli (lineari) o da fasi (di un
	      ciclo): dei primi sono memorizzati l'istante di inizio e l'istante di
	      fine mentre delle seconde, il cui codice è progressivo, il nome.
	\item \textbf{Frasi relative ai luoghi:}
	      I luoghi sono caratterizzati da nome e coordinate su una data mappa.
	\item \textbf{Frasi relative alle mappe:}
	      Le mappe sono descritte da nome, immagine ed estensione.
	\item \textbf{Frasi relative agli eventi:}
	      Gli eventi sono caratterizzati da veridicità, agenti coinvolti, luogo e
	      intervallo. Uno o più agenti, per un dato intervallo di tempo, possono
	      pensare che un evento sia avvenuto.
	\item \textbf{Frasi relative alle unità narrative:}
	      Le unità narrative sono descritte da indice, titolo e intervallo narrato.
\end{itemize}

\subsection{Specifica operazioni}

\begin{enumerate}
	\item Inserire un nuovo agente ()
	\item Inserire un nuovo intervallo () ???
	\item Inserire una nuova fase () ???
	\item Inserire un nuovo luogo ()
	\item Inserire una nuova mappa ()
	\item Inserire un nuovo evento()
	\item Inserire una nuova unità narrativa ()
	\item Visualizzare ()
	\item Visualizzare ()
	\item Visualizzare ()
	\item Visualizzare ()
	\item Visualizzare ()
	\item Visualizzare ()
\end{enumerate}


\section{Progettazione concettuale}

\subsection{Identificazione delle entità e relazioni (bottom-up) ???}
Sono state identificate le seguenti entità: personaggio, alias, intervallo,
fase, luogo, mappa, evento e unità narrativa.
Le prime due entità sono comprese dall'entità agente, ???
\subsection{Un primo scheletro dello schema (top-down)}
A un primo livello di astrazione è stato concepito un primo scheletro di schema
concettuale:

dove Chi ???, Quando ???, Dove ??? e Credenza ???
\subsection{Sviluppo delle componenti dello scheletro (inside-out)}

\subsection{Unione delle componenti nello schema finale ridotto}

\subsection{Dizionario dei dati}

\begin{tabular}{ |c|c|c|c| }
	\hline
	\textbf{Nome entità} & \textbf{Descrizione} & \textbf{Attributi} & \textbf{Identificatore} \\
	\hline
	cell5                & cell6                & cell7              & cell8                   \\
	\hline
	cell9                & cell10               & cell11             & cell12                  \\
	\hline
\end{tabular}

\begin{tabular}{ |c|c|c|c| }
	\hline
	\textbf{Nome relazione} & \textbf{Descrizione} & \textbf{Entità coinvolte} & \textbf{Attributi} \\
	\hline
	cell5                   & cell6                & cell7                     & cell8              \\
	\hline
	cell9                   & cell10               & cell11                    & cell12             \\
	\hline
\end{tabular}

\subsection{Regole}

\begin{tabular}{ |c| }
	\hline
	\textbf{Regole di vincolo}     \\
	\hline
	cell5                          \\
	\hline
	cell9                          \\
	\hline
	\textbf{Regole di derivazione} \\
	\hline
	cell5                          \\
	\hline
	cell9                          \\
	\hline
\end{tabular}

\section{Progettazione logica}

\subsection{Tavole dei volumi e delle operazioni}
\textbf{Tavola dei volumi:}

\begin{tabular}{ |c|c|c| }
	\hline
	\textbf{Concetto} & \textbf{Tipo} & \textbf{Volume} \\
	\hline
	cell5             & cell6         & cell7           \\
	\hline
	cell9             & cell10        & cell11          \\
	\hline
\end{tabular}

\textbf{Tavola delle operazioni:}

\begin{tabular}{ |c|c| }
	\hline
	\textbf{Operazione} & \textbf{Frequenza} \\
	\hline
	cell5               & cell6              \\
	\hline
	cell9               & cell10             \\
	\hline
\end{tabular}

\subsection{Ristrutturazione dello schema concettuale}

\subsection{Normalizzazione}
Analizzando lo schema concettuale si nota che tutte le associazioni presenti
sono in forma normale di Boyce e Codd perché ???

\textbf{Entità:}

\begin{tabular}{ |c|c| }
	\hline
	\textbf{Nome entità} & \textbf{Commento} \\
	\hline
	cell5                & cell6             \\
	\hline
	cell9                & cell10            \\
	\hline
\end{tabular}

\subsection{Traduzione verso il modello relazionale}

\begin{tabular}{ |c|c| }
	\hline
	\textbf{Entità-Relazione} & \textbf{Traduzione} \\
	\hline
	cell5                     & cell6               \\
	\hline
	cell9                     & cell10              \\
	\hline
\end{tabular}

\begin{tabular}{ |c|c| }
	\hline
	\textbf{Traduzione} & \textbf{Vincoli di riferimento} \\
	\hline
	cell5               & cell6                           \\
	\hline
	cell9               & cell10                          \\
	\hline
\end{tabular}


\section{Codifica SQL}

\subsection{Definizione dello schema}

\subsection{Codifica delle operazioni}

\section{Testing}

\end{document}
